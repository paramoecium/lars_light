% IEEE standard conference template; to be used with:
%   spconf.sty  - LaTeX style file, and
%   IEEEbib.bst - IEEE bibliography style file .
% --------------------------------------------------------------------------

\documentclass[letterpaper]{article}
\usepackage{spconf,amsmath,amssymb,graphicx}
\usepackage{algorithm}
\usepackage{algpseudocode}
\usepackage{color}
\usepackage{pifont}
\usepackage{subcaption}

% Example definitions.
% --------------------
% nice symbols for real and complex numbers
\newcommand{\R}[0]{\mathbb{R}}
\newcommand{\C}[0]{\mathbb{C}}
\newcommand{\lc}[1]{\ding{110}\ding{43}\textcolor{blue}{LC: #1}}
\newcommand{\yt}[1]{\ding{110}\ding{43}\textcolor{blue}{YT: #1}}
\newcommand{\MP}[1]{\ding{110}\ding{43}\textcolor{blue}{Markus Püschel: #1}}

% bold paragraph titles
\newcommand{\mypar}[1]{{\bf #1.}}

% Title.
% ------
\title{An Effective Optimization on Least Angle Regression}
%
% Single address.
% ---------------
% \name{Markus P\"uschel\thanks{The author thanks Jelena Kovacevic. This paper
% is a modified version of the template she used in her class.}} 
% \address{Department of Computer Science\\ ETH Z\"urich\\Z\"urich, Switzerland}

% For example:
% ------------
%\address{School\\
%		 Department\\
%		 Address}
%
% Two addresses (uncomment and modify for two-address case).
% ----------------------------------------------------------
\twoauthors
 {Li Chang}
 {Department of Computer Science\\ ETH Z\"urich\\Z\"urich, Switzerland}
% 		 {School A-B\\
% 		 Department A-B\\
% 		 Address A-B}
 {Yu-Chen Tsai}
 {Department of Computer Science\\ ETH Z\"urich\\Z\"urich, Switzerland}
% 		 {School C-D\\
% 		 Department C-D\\
% 		 Address C-D}


\begin{document}
%\ninept
%
\maketitle
%

% The hard page limit is 6 pages in this style. Do not reduce font size
% or use other tricks to squeeze. This pdf is formatted in the American letter format, so may look a bit strange when printed out.

\begin{abstract}

Dictionary learning is a technique that learns a basis set and applies it to a specific data.
Such approach has been proven to be very effective for signal processing and classification problems.
Recent paper~\cite{Mairal:2010} has proposed an algorithm that can processed large data set using the idea of \emph{online learning}.
The backbone of this dictionary learning algorithm is Least Angle Regression (LARS) that reconstructs the specific data using the basis in dictionary.

While training a model, the shorter time an iteration needs, the more data can be consumed and the more general the trained model can be.
Therefore, in this project, we aim to optimize the LARS algorithm which is not only the main component of online dictionary learning but is also widely used for applications related to feature selection.

We started with implementing a baseline version from scratch, referencing a more general implementation in C++ and CBLAS \cite{larscpp}.
We optimize the baseline implementation by reusing computations, improving locality, and applying instruction-level parallelism and vectorization. 
We achieved a 7.2 times speed-up in total and increase the performance from 0.57 flops/cycle to 4.10 flops/cycle.



% Describe in concise words what you do, why you do it (not necessarily
% in this order), and the main result.  The abstract has to be
% self-contained and readable for a person in the general area. You
% should write the abstract last.a
\end{abstract}

\section{Introduction}
\label{sec:introduction}
% \mypar{Motivation}
% What is online dictionary learning, why is this important.
% mention we only optimize lars?
% We optimize LARS, why we optimize LARS
% What have we optimized?
% One sentence result on how we improve it.
% Give a small description on what each section is doing
% \mypar{Related work?}


% \mypar{Online Sparse Dictionary Learning}
% % include the psuedo-code from the original paper?
Dictionary learning has been used widely in machine learning, signal processing and image processing.
In sparse dictionary learning, the goal is to approximate a given signal using only a few basis elements from the learned dictionary. More specifically, given a bunch of signals $Y$ in $\mathbb{R}^d$ and a dictionary $X$ in $\mathbb{R}^{d \times k}$, with k columns referred to as atoms, the algorithm is meant for finding a linear combination of most representative atoms from the dictionary.

To handle very large training sets and dynamic training data changing over time, Mairal et al. proposes an online learning algorithm for dictionary learning\cite{Mairal:2010}. In each iteration $i$ of the algorithm, sparse coding is used to compute the sparse representation $\beta_i$ of a random sample $y$ in $\mathbb{R}^d$ from a distribution $p(y)$. $\beta_i$ is then used to update the dictionary $X$.

% \mypar{Least Angle Regression (LARS)}
LARS is one of the well-known sparse coding algorithms, developed by Bradley Efron et al.\cite{Efron:04} that is less greedy and more computationally efficient. Given a collection of column vectors as dictionary $X$, LARS always picks the columns of $X$ most correlated to current residual and walk along their equiangular direction until some other unused column has as much correlation with the current residual.

Since LARS is a significant part of the dictionary learning and is also widely applicable in feature selection applications, the goal of this paper is to proposed optimization methods that are applicable to the implementation of LARS ans its variants, and justify these optimization methods do speed up the entire procedure and enhance the performance.

In the following sections, we first formally define LARS, introduce incremental Cholesky decomposition and shows a detail cost analysis on the entire procedure in Section~\ref{sec:background}.
Then we present the optimization we applied in Section~\ref{sec:method} and show the result in Section~\ref{sec:experiment}. At last, we summarize our work and list future works in Section~\ref{sec:conclusions}.






%  Do not start the introduction with the abstract or a slightly modified
%  version. It follows a possible structure of the introduction. 
%  Note that the structure can be modified, but the
%  content should be the same. Introduction and abstract should fill at most the first page, better less.
 
%  \mypar{Motivation} The first task is to motivate what you do.  You can
%  start general and zoom in one the specific problem you consider.  In
%  the process you should have explained to the reader: what you are doing,
%  why you are doing, why it is important (order is usually reversed).
 
%  For example, if my result is the fastest DFT implementation ever, one
%  could roughly go as follows. First explain why the DFT is important
%  (used everywhere with a few examples) and why performance matters (large datasets,
%  realtime). Then explain that fast implementations are very hard and
%  expensive to get (memory hierarchy, vector, parallel). 
 
%  Now you state what you do in this paper. In our example: 
%  presenting a DFT implementation that is
%  faster for some sizes than all the other ones.
 
%  \mypar{Related work} Next, you have to give a brief overview of
%  related work. For a paper like this, anywhere between 2 and 8
%  references. Briefly explain what they do. In the end contrast to what
%  you do to make now precisely clear what your contribution is.


\section{Background}
\label{sec:background}

In this section, we provide necessary information to understand the implementation of the LARS algorithm.
We first formally define LARS and explain the algorithm that solve the given formulas in Section~\ref{ssec:lars}
We then introduce Incremental Cholesky decomposition in Section~\ref{ssec:cholesky} and formulate how we calculate the cost of the program in Section~\ref{ssec:cost-analysis}.
At last, in Tabel~\ref{tab:cost} we present the exact number of operations of the entire program.

\subsection{A Formal Definition of LARS}
\label{ssec:lars}
A L1-regularized linear regression problem and its dual form, solved by LARS, is described respectively as follows:
$$
min_\beta{||y-X\beta||^2_2} \ s.t\ ||\beta||_1 \leq l
$$
$$
min_\beta{||y-X\beta||^2_2} + \lambda||\beta||_1
$$

A more detailed explanations of how LARS produces a piece-wise linear solution path is described in Algorithm~\ref{alg:lars}.
Initially, the current residual is equal to the input $y$. 
The correlation between the current approximation and the current residual is initialized (line~\ref{alg:lars:initailize_correlation}).
In each iteration, the correlation is updated and the most correlated basis to current residual is added into the active set $A$ (line~\ref{alg:lars:get_active_idx} - \ref{alg:lars:get_active_idx_end}).
The unit vector of the equiangular direction, $u_A$, is computed in line~\ref{alg:lars:cholesky} - \ref{alg:lars:get_a}, and the distance to walk is computed at line~\ref{alg:lars:get_gamma}.
The current approximation of $y$ is then updated at line~\ref{alg:lars:update_beta}.
The \texttt{while} loop continues until the $\hat{\lambda}$, computed at line~\ref{alg:lars:compute_lambda} is greater than the given parameter $\lambda$.


% Insert the algorithm
\begin{algorithm}
	\caption{Compute sum of integers in array}
	\label{alg:lars}
	\begin{algorithmic}[1]
		\Procedure{LARS}{$X, y, \lambda$}
		    \State $\hat{c} = X^T y$, $\hat{\beta} \gets 0$ \label{alg:lars:initailize_correlation}
		    \While {\textsc{Compute\_$\lambda$()} $< \lambda$ }
		      % Get_active_idx
		      \State $\hat{c} \gets \hat{c} - X^T \hat{\beta}$ \label{alg:lars:get_active_idx}
		      \State $\hat{C} \gets \max_j{|\hat{c}_j|}$
		      , $A \gets \{j, |\hat{c}_j| = \hat{C} \}$
		      \State $s_A \gets \{sign(\hat{c}_j), j \in A\}$ \label{alg:lars:get_active_idx_end}
		      
		      % Cholesky
		      \State $G_A \gets X_A^T X_A$ \label{alg:lars:cholesky}
		      \State $w_A \gets G_A^{-1} s_A$
		      \label{alg:lars:inversion}
		      \State $A_A \gets \sqrt{1_A^T w_A}$ \label{alg:lars:cholesky_end}
		      
		      % Get_U
		      \State $u_A \gets A_A X_A w_A$ \label{alg:lars:get_u}
		      
		      % Get_A
		      \State $a \gets X^T u_A$ \label{alg:lars:get_a}

		      % Get_gamma
		      \State $\hat{\gamma} = \min_{j \in A^c}^+ \Big\{ \frac{\hat{C} - \hat{c_j}}{A_A - a_j},  \frac{\hat{C} + \hat{c_j}}{A_A + a_j}\Big\}$ \label{alg:lars:get_gamma}
		      
		      % Update_beta
		      \State $\hat{\beta} \gets \hat{\beta} + \hat{\gamma} u_A$ \label{alg:lars:update_beta}
		    
		    \EndWhile
            \State Return $\hat{\beta}$
		\EndProcedure
		
		\Procedure{Compute\_$\lambda$}{}\label{alg:lars:compute_lambda}
		    \State $\Lambda = X_A^T ( X_A \hat{\beta} - y)$
    		\State Return $\max_{\lambda_i \in \Lambda} \big\{ |\lambda_i| \big\}$ 
		\EndProcedure
	\end{algorithmic}
\end{algorithm}


\subsection{Incremental Cholesky}
\label{ssec:cholesky}
%\mypar{baseline implementation}
$w_A$ (line~\ref{alg:lars:inversion}, Algorithm~\ref{alg:lars} is the weighting function that composes the new direction to march along.
$w_A$ is decided by the inversion of the correlation matrix of all the current active basis, $G_A$.
Since the active set grows by one in every iteration, $G_A$ has to be recalculated according to the current active set.
Instead of solving the entire inverse matrix, which is $O(n^3)$ in complexity, in every iteration,
we only update the inverse matrix according to the newly added basis.
Such technique is called Incremental Cholesky, and is only of $O(N^2)$ complexity for each update.
We separate Incremental Cholesky into two parts: 

\mypar{Update Cholesky solver}
Suppose we have a correlation matrix $X_{A_{t-1}}^T X_{A_{t-1}}$, in iteration $t$, to update the correlation matrix with the new basis $v$, we only add a new column and a new row to the previous correlation matrix, getting the new correlation matrix:
\[
G_{A_t} = 
\begin{bmatrix}
X_{A_{t-1}}^T X_{A_{t-1}}   &   X_{A_{t-1}}^T v \\
v^T X_{A_{t-1}}             &   \sqrt{v^T v}
\end{bmatrix}
\].
We then update the lower triangle of $G_A$ with:
\[
L_t = 
\begin{bmatrix}
L_{t-1}   &    0 \\
v^T X_{A_{t-1}} (L_{t-1}^{-1})^T  &  \sqrt{v^T v - |v^T X_A (L_{t-1}^-1)^T | ^2}
\end{bmatrix}
\]
To compute $v^T X_{A_{t-1}} (L_{t-1}^{-1})^T$, we solve $w$ for
$$
L_{t-1} w = X_{A_{t-1}} v
$$ with Gaussian elimination. 

\mypar{backsolve the target}
Inverting correlation matrix $G_{A_t}$ is inverting the corresponding Cholesky decomposition $LL^T$, which can be done by solving a lower triangular system and a upper triangular system sequentially. 



\subsection{Cost Analysis}
\label{ssec:cost-analysis}
% Define cost measure
Each math function (add, mult, div, sqrt, etc.) is associated with a specific number of floating point operations.
We derived the corresponding floating point operations according to the ratio of its throughput to the throughput of floating point addition.
For example, a division of two doubles is counted as 16 floating point operations, because the throughput of \texttt{\_mm256\_div\_pd} is $1/8$ while \texttt{\_mm256\_add\_pd} is 2\cite{Intrinsics}.
The relative floating point operation counts of all code segments in the algorithm is listed in Table~\ref{tab:flop_def}.
 
\begin{table}
\centering
\begin{tabular}{|c||c|c|c|c|c|c|c|}
\hline
operation & cmp & add & mul & fma & div & sqrt & abs \\ \hline
flop count & 1 & 1 & 1 & 1 & 16 & 24 & 1.5 \\ \hline
\end{tabular}
\caption{Relative flop count of operations.}
\label{tab:flop_def}
\end{table}

%% Description for the table.
Since we are solving sparse and over-complete representation of the input, dictionary size $k$ is usually larger than signal dimension $d$. The number of iterations in LARS depends on regularization parameters $\lambda$ and the value of target signals. For simplicity of analysis, we set $k = 2d$ and $\lambda = 0$, so that the algorithm always terminates on the $d^{th}$ iteration. 

For optimization purposes, we counted the exact floating point operations within different parts of the algorithm LARS as shown in Table~\ref{tab:cost}. With Table~\ref{tab:flop_def} and Table~\ref{tab:cost}, we can calculate the floating point operation counts of each code segment.


\begin{table*}[ht!]
\centering
\begin{tabular}{|c || c | c | c | c | c | c | c | c |}
\hline
 Code Segment & Line & cmp & add & mul & fma & div & sqrt & abs \\
\hline\hline
\textsc{Init\_Correlation} & 2 & 0 & 0 & 0 & $2D^2$ & 0 & 0 & 0 \\ 
\textsc{Find\_Active\_Idx} & 4-6 & $7D^2+D$ & 0 & 0 & 0 & 0 & 0 & $\frac{D(D+1)}{2}$ \\
\textsc{Fused\_Cholesky} & 7-9 & $D$ & $\frac{D(D+1)}{2}$ & 0 & $\frac{D(D+1)(8D-2)}{6}$ & $\frac{D(D+1)}{2}$ & $2D$ & $\frac{D(D+1)}{2}$ \\
\textsc{Compute\_U} & 10 & 0 & 0 & $\frac{D(D+1)}{2}$ & $\frac{D^2(D+1)}{2}$ & 0 & 0 & 0  \\
\textsc{Compute\_A} & 11 & 0 & 0 & 0 & $2D^3$ & 0 & 0 & 0  \\
\textsc{Compute\_$\gamma$} & 12 & $4D^2+3D$ & $2D^2+2D$ & 0 & 0 & $D^2+2D$ & 0 & 0  \\
\textsc{Update\_$\beta$} & 13 & 0 & 0 & 0 & $\frac{D(D+1)}{2}$ & 0 & 0 & 0  \\
\textsc{Compute\_$\lambda$} & 18 & $2D^2$ & 0 & 0 & $\frac{D^2(D+3)}{2}$ & 0 & 0 & $2D^2$  \\
\hline
\textsc{Total} & * & $13D^2+4D$ & $\frac{5D(D+1)}{2}$ & $\frac{D(D+1)}{2}$ & $\frac{D(29D^2+42D+1)}{6}$ & $\frac{D(3D+5)}{2}$ & $2D$ & $D(3D+1)$  \\
\hline
\end{tabular}
\caption{Cost analysis on each code segments of Algorithm~\ref{alg:lars}. $D$ is the dimension of the target signal.}
\label{tab:cost}
\end{table*}

% Asymptotic complexity






%  Give a short, self-contained summary of necessary
%  background information. For example, assume you present an
%  implementation of FFT algorithms. You could organize into DFT
%  definition, FFTs considered, and cost analysis. The goal of the
%  background section is to make the paper self-contained for an audience
%  as large as possible. As in every section
%  you start with a very brief overview of the section. Here it could be as follows: In this section 
%  we formally define the discrete Fourier transform, introduce the algorithms we use
%  and perform a cost analysis.
%  
%  \mypar{Discrete Fourier Transform}
%  Precisely define the transform so I understand it even if I have never
%  seen it before.
%  
%  \mypar{Fast Fourier Transforms}
%  Explain the algorithm you use.
%  
%  \mypar{Cost Analysis}
%  First define you cost measure (what you count) and then compute the
%  cost. Ideally precisely, at least asymptotically. In the latter case you will 
%  need to instrument your code to count
%  the operations so you can create a performance plot.
%  
%  Also state what is
%  known about the complexity (asymptotic usually) 
%  about your problem (including citations).
%  
%  Don't talk about "the complexity of the algorithm.'' It's incorrect,
%  remember (Lecture 2)?

\section{Optimization on LARS}
\label{sec:method}

% First introduce what we've done
% Show cycles pie chart and tell why we put our focus on the four fat parts
In this section, we fist briefly explain the baseline implementation of the LARS algorithm and state assumptions made to simplify the code. 
We then elaborate on optimization on most time-consuming parts of the algorithm and go through some general optimization.


% \subsection{Baseline implementation}
% General overview on baseline implementation
% Assumptions
% How we segment our code and why

Since each step of the algorithm requires results from the previous step, we segmented our code into several segments accordingly, as mentioned in Table~\ref{tab:cost}.

The baseline is a straight-forward implementation of Algorithm~\ref{alg:lars} in C++ from scratch. The dictionary $X$, the largest array of the algorithm, is store in a column-majored format before any optimization. For implementation simplicity, we assume both dimensions of the dictionary, $d$ and $k$, are power of two and divisible by 16. Figure~\ref{fig:pie_before} shows that \textsc{Compute\_Lambda}, \textsc{Compute\_A}, and \textsc{Cholesky} are the most expensive components. Optimizing these parts therefore has the highest priority. 

\begin{figure}
\centering
  \includegraphics[scale=0.29]{./pic/pie_before.png}
  \caption{Run-time before optimization}
  \label{fig:pie_before}
\end{figure}

\subsection{Reuse previous computation}
Most parts of our algorithm are computational bounded. Therefore, we try to find redundant computations and reuse the result of previous computation.
Possible methods proposed are storing previous computed value and reuse previous computation according to mathematical implications.

\mypar{Trade computation with memory}
In the beginning of each iteration, LARS check if the current approximation hit the regularize parameter $\lambda$. The equation used to compute $\lambda$ is shown in line~\ref{alg:lars:compute_lambda} in Algorithm~\ref{alg:lars}.
It is clear that 
\begin{equation}
    \begin{split}
        \Lambda &= X_A^T ( X_A \hat{\beta} - y) \\
                &= X_A^T X_A \hat{\beta} - X_A^T y \\
                &= G_A \hat{\beta} - X_A^T y
    \end{split}
\end{equation}
Since $G_A$ is calculated when updating the Cholesky solver (line~\ref{alg:lars:cholesky}, Algorithm~\ref{alg:lars}), we store this value in a new array. This requires more memory for variables since, previously, solving an incremental Cholesky system only need the last row of $G_A$ in each iteration.
This method not only save computation time but also reduce accessed memory.
It decreases the number of memory access from $\mathbb{R}^{|A| \times k}$ to $\mathbb{R}^{|A| \times |A|}$, where $|A|$ is the size of the active set.
We achieve significant speed-up in this segment with this modification.

\mypar{Remove computation using mathematical implication}
In the baseline version of the implementation,
the target value $s_A$ is re-computed at every iteration before fed into the Cholkesy solver(line~\ref{alg:lars:get_active_idx_end}, Algorithm~\ref{alg:lars}).
The reason to do so is that the correlation between the current residual and the target is updated according to the current approximation in each iteration (line~\ref{alg:lars:get_active_idx}, Algorithm~\ref{alg:lars}).
Since the algorithm assures convergence, it implies that the sign of the correlation does not change once it is added into the active set.
Therefore, instead of recomputing the sign of each entry in every iteration, only the sign of the  added entry is calculated.

\subsection{Improve locality}
Since there are a lot of memory access throughout the algorithm.
We proposed several effective methods to take advantage of CPU caching.

\mypar{Compress stored data}
To keep track of the Cholesky decomposition of the correlation matrix $G$, we record the lower triangular matrix $L$, where $G = LL^T$. Since all the entries of $L$ above the main diagonal are zero and never used, instead of storing $L$ as a square matrix, we store only the lower triangle as a flattened one dimensional array. Even though in this case we lose much data alignment, it turns out to bring more benefits by fitting more of $L$ into cache.

The gram matrix $G_A$ also appears in the optimized version for computing $\lambda$ in the beginning of each iteration. Instead of storing the entire gram matrix, we only stored the lower triangular matrix using the compressed form.

\mypar{Switch to column orientation for $L^T$}
The Cholesky updating step solves a system of $L$ while the Cholesky inverting step solves a system of L and then a system of $L^T$. To solve two systems of $L$ and one of $L^T$, the program needs to go through every entry in $L$ three times. We don't store $L^T$ explicitly. Instead, while solving a system of $L^T$, the program accesses $L$ in the required order. As a Gaussian elimination can be conducted in either row orientation or column orientation, we pick the proper orientation for the system of $L$ and the system of $L^T$ respectively. 

\mypar{Fuse $L$ solving}
Furthermore, we fuse the Cholesky updating step with the Cholesky inverting step for both of them are solving a system of $L$. We can solve the system with different right-hand sides simultaneously. This way we only need to go through $L$ twice, one for the systems of $L$ and the other for the system of $L^T$.

\begin{figure}
\begin{subfigure}{.23\textwidth}
  \includegraphics[scale=0.2]{./pic/block_cholesky.png}
  \caption{Block triangular solve}
  \label{fig:block_cholesky}
\end{subfigure}%
\begin{subfigure}{.22\textwidth}
  \includegraphics[scale=0.155]{./pic/fuse_cholesky.png}
  \caption{Fused triangular solve}
  \label{fig:fuse_cholesky}
\end{subfigure}
\caption{Optimization of Cholesky decomposition}
\label{fig:test}
\end{figure}

While reusing the value of the gram matrix $G_A$ to compute $\lambda$, both the lower triangular matrix of $G_A$ and its transpose are needed for computation: $G_A \hat{\beta} = L_{G_A} L_{G_A}^T \hat{\beta}$. We fuse the two multiplication by accessing each entry of $L_{G_A}$, the lower triangular matrix of $G_A$, only once.

\mypar{Blocking} All calculations done on the triangular solver in Cholesky is blocked with block size 128, so that each block fits in the L2 cache. 
Similarly, when computing $u_A$ and $a$ (line~\ref{alg:lars:get_u} and line~\ref{alg:lars:get_a} in Algorithm~\ref{alg:lars}), the matrices and vectors are also blocked to size 512 to improve locality.
Computing $\lambda$ (line~\ref{alg:lars:compute_lambda} in Algorithm~\ref{alg:lars}) is also blocked and is set to a block size of 16 after several testings on the optimal block size.


% compute vector A, access X in order
\mypar{Access large data structure in order}
Since $X_A$ is only a permuted subset of columns in $X$, we does not really record every entry in $X_A$. Instead, we only record the order of the number of column in $X$ that is added into $X_A$ in our baseline version.

To get the direction to move on, it is necessary to compute $u_A = A_A X_A w_A$ and $a = X^T u_A$, as shown in line~\ref{alg:lars:get_u} in Algorithm~\ref{alg:lars}.
The problem here is that to access $X_A$ in order, we will be jumping from rows to rows in $X$.
Since $X$ is large, it might cause multiple larger level cache misses.

To prevent this from happening, the best way is to sort the active set $A$ so that you access everything throughout the algorithm in order.
However, since we use incremental cholesky in our algorithm, we could not directly sort the active set in every iteration.
Therefore, we only attempt to access large data chunk, $X$, in order, while accessing smaller component $u_A$ incontinuously.

\subsection{Instruction-level parallelism and vectorization}
After improving the locality we vectorize most of our code with Intel X86 AVX2 intrinsics.
With the specific operations provided by the library, we are able to indicate FMA operations,
unroll for loops and remove if branches.

\mypar{unroll and scalar replacement}
To avoid blocking and increase instruction-level parallelism, we also implemented unrolling plus scalar replacement in for loops.
We also tested unrolling in each segment of the code until we get the best unrolling factor.

\mypar{remove if branches}
To calculate the absolute value, the sign value, checking whether a column is already activated and getting only values that satisfy some constraints create expensive if branches throughout the algorithm.
We therefore replace these if branches by combining \texttt{blendv} and \texttt{cmp} to get rid of the overhead caused by if branches.







% Now comes the ``beef'' of the paper, where you explain what you
% did. Again, organize it in paragraphs with titles. As in every section
% you start with a very brief overview of the section.

% For this class, explain all the optimizations you performed. This mean, you first very briefly
% explain the baseline implementation, then go through locality and other optimizations, 
% and finally SSE (every project will be slightly different of course). Show or mention 
% relevant analysis or assumptions. A few examples: 1) Profiling may lead you to 
% optimize one part first; 2) bandwidth plus data transfer analysis may show that 
% it is memory bound; 3) it may be too hard to implement the algorithm in full 
% generality: make assumptions and state them (e.g., we assume $n$ is divisible 
% by 4; or, we consider only one type of input image); 4) explain how certain data 
% accesses have poor locality. Generally, any type of analysis adds value to your work.

% As important as the final results is to show that you took a structured, organized 
% approach to the optimization and that you explain why you did what you did.

% Mention and cite any external resources including library or other code.

% Good visuals or even brief code snippets to illustrate what you did are good. 
% Pasting large amounts of code to fill the space is not good.



\section{Experimental Results}
\label{sec:experiment}

In this section, we evaluate the optimization strategies proposed in Section~\ref{sec:method}.
We detail the settings of the machine that we performed our test on, and then
present the results of our optimization.

\subsection{Experimental setup}

All performance experiments are conducted on Red Hat Enterprise Linux 7 running on a machine with the specifications as shown in Table~\ref{tab:cpu-info}.
% CPU:Intel i7-6700 CPU @ 3.40GHz (skylake)\\
% L1 data cache,    	line size 64,  8-ways,	64 sets, size 32k\\
% L1 instruction cache, line size 64,  8-ways,	64 sets, size 32k\\
% L2 unified cache, 	line size 64,  4-ways,  1024 sets, size 256k\\
% L3 unified cache, 	line size 64, 16-ways,  8192 sets, size 8192k
% \lc{use table for this?}
\begin{table}
\begin{center}
\begin{tabular}{ | c | c | c | c | c | }
%  \hline
%  \multicolumn{5}{|l|}{Intel i7-6700 CPU @ 3.40GHz (skylake)}\\
%  \hline
 \hline
     & line size & \# ways & \# sets  & size\\  \hline
 L1 data cache        & 64 &  8 &   64 & 32K \\  \hline
 L1 instruction cache & 64 &  8 &   64 & 32K \\  \hline
 L2 unified cache     & 64 &  4 & 1024 & 256K \\ \hline
 L3 unified cache     & 64 & 16 & 8192 & 8M \\  
 \hline
\end{tabular}
\caption{Specification of Intel i7-6700 CPU @ 3.40GHz (skylake).\cite{Intel_i7-6700}}
\label{tab:cpu-info}
\end{center}
\end{table}

The maximal memory bandwidth advertised is 34.1 GB/s, but in practice it's guaranteed to be less than that. We hence use \textit{bandwidth}\cite{bandwidth} to measure sustained memory bandwidth for different access patterns. From the benchmark result in Figure~\ref{fig:mem-bw}, the memory bandwidth of random read and write are 21.2 GB/s and 3.10 GB/s respectively for data of size 512 MB. We thus take these two bandwidth as the memory bounds as the upper-bound of the memory in our roofline model.

The intersection of the peak performance bound and the memory bandwidth bound is at $I = \pi/\beta$. For scalar code, the intersection is between 0.64 and 4.38 flop/byte. As for vectorized code, the intersection is between 2.56 and 17.55 flop/byte.

We use Linux \texttt{perf} command \cite{perf} to monitor hardware event such as cache miss and number of load/store invocation. The number of last-level cache (LLC) miss can give us an idea of the data traffic between CPU and main memory.

%% should we talk about how we segment our code here?

The compilers we use for all the statistics in this paper are Intel C/C++ Compiler (\textit{icc 14.0.0.20131008}) For the baseline, the compiler flag is \textit{-O3 -ansi-alias -no-vec -unroll=0}. As for the our best optimized version, we set the flag to \textit{-O3 -std=c++11 -xHost -ansi-alias -unroll=4 -march=core-avx2 -auto-ilp32}.

% \begin{center}
% \begin{tabular}{ | c | c | c | }\label{tab:cpu-info}
%  \hline
%  cell1 & cell2 & cell3 \\
%  \hline
%  cell4 & cell5 & cell6 \\  
%  cell7 & cell8 & cell9 \\
%  \hline
% \end{tabular}
% \caption{}
% \end{center}


%% Results

\subsection{Results}

Figure~\ref{fig:performance} shows the performance gain from all the optimization techniques we adopt. With proper unrolling and vectorize compiler flags, the performance is improved by 3.7 times, from 0.57 flops/cycle to 2.12 flops/cycle. With all the other approaches mentioned in Section~\ref{sec:method} that improves data locality and reduce computation, we further enhanced the performance to 4.10 flops/cycle, which gives us 7.2 times speed-up in total comparing to the baseline version. The run time proportion after optimization is shown in Figure~\ref{fig:pie_after} 

\begin{figure}
\centering
  \includegraphics[scale=0.26]{./pic/pie_after.png}
  \caption{Run time after optimization}
  \label{fig:pie_after}
\end{figure}

\begin{figure}[h]
\centering
\includegraphics[width=0.5\textwidth]{./pic/performance.png}
\caption{Performance Comparison}
\label{fig:performance}
\end{figure}


\begin{figure*}
\centering
\includegraphics[scale=0.28]{./pic/bandwidth.jpg}
\caption{Measured memory bandwidth}
\label{fig:mem-bw}
\end{figure*}

As the input size increases, the overhead of optimization, such as extra branches for corner case of blocking and unrolling, become insignificant as the benefit grows faster. On the other hand, the performance and the speed-up gain drop dramatically on input sizes larger than 1024. One reason might be that for input size of 1024, the allocated memory space exceed the 8MB physical memory space. Then the run time is dominated by the access time of hard disk.

The memory traffic is estimated as: 
$$
\# LLC\ misses \cdot LLC\ line\ size
$$
As argued by Georg Ofenbeck et al.\cite{ofenbeck2014applying} the actual traffic might be more than, even twice, the amount of observed LLC cache misses, depending on caching mechanism. Even though, it's still good upper bound of of memory traffic. Combining the measured memory traffic, measured run time cycles, and calculated floating point operation counts, we draw the roofline plot and examine the limitation of optimizing this algorithm.

\begin{figure}[h]
\centering
\includegraphics[width=0.5\textwidth]{./pic/roofline.png}
\caption{Roofline plot of baseline and optimized program.}
\label{fig:roofline}
\end{figure}

From the roofline plot in Figure~\ref{fig:roofline}, we know that the larger the input size is (larger than 256), the more likely the algorithm is memory bounded.




% Here you evaluate your work using experiments. You start again with a
% very short summary of the section. The typical structure follows.

% \mypar{Experimental setup} Specify the platform (processor, frequency, cache sizes)
% as well as the compiler, version, and flags used. I strongly recommend that you 
% play with optimization flags and consider also icc for additional potential speedup.

% Then explain what input you used and what range of sizes. The idea is to give 
% enough information so the experiments are reproducible by somebody else on his or her code.

% \mypar{Flags}
% -fma or/Qfma. If the instructions exist on the target processor, the compiler generates fused multiply-add (FMA) instructions.

% \mypar{Results}
% Next divide the experiments into classes, one paragraph for each. In the simplest
% case you have one plot that has the size on the x-axis and the performance on the
% y-axis. The plot will contain several lines, one for each relevant code version.
% Discuss the plot and extract the overall performance gain from baseline to best 
% code. Also state the percentage of peak performance for the best code. Note that
% the peak may change depending on the situation. For example, if you only do 
% additions it would be 12 Gflop/s
% on one core with 3 Ghz and SSE and single precision floating point.

% Do not put two performance lines into the same plot if the operations count 
% changed significantly (that's apples and oranges). In that case first perform 
% the optimizations that reduce op count and report the runtime gain in a plot. 
% Then continue to optimize the best version and show performance plots.

% {\bf You should}
% \begin{itemize}
% \item Follow the guide to benchmarking presented in class, in particular
% \item very readable, attractive plots (do 1 column, not 2 column plots
% for this class), proper readable font size. An example is below (of course you 
% can have a different style),
% \item every plot answers a question, which you pose and extract the
% answer from the plot in its discussion
% \end{itemize}
% Every plot should be discussed (what does it show, which statements do
% you extract).


\section{Conclusions}
\label{sec:conclusions}
In this paper, we proposed several optimization strategies that can be applied to the LARS algorithm and give a summary on their performance.
Besides optimization that can also be done by the compilers, we also proposed specific optimization according to its mathematical implication and data access pattern.
We achieved an overall $7.2\times$ speed-up, reaching 4.10 flops/cycle, in performance against the baseline implementation without optimization flags (0.57 flops/cycle) and a $2.0\times$ speed-up from a compiler optimized version (2.12 flops/cycle).






% Here you need to briefly summarize what you did and why this is
% important. {\em Do not take the abstract} and put it in the past
% tense. Remember, now the reader has (hopefully) read the paper, so it
% is a very different situation from the abstract. Try to highlight
% important results and say the things you really want to get across
% (e.g., the results show that we are within 2x of the optimal performance ... 
% Even though we only considered the DFT, our optimization
% techniques should be also applicable ....) You can also formulate next
% steps if you want. Be brief.

% References should be produced using the bibtex program from suitable
% BiBTeX files (here: bibl_conf). The IEEEbib.bst bibliography
% style file from IEEE produces unsorted bibliography list.
% -------------------------------------------------------------------------
\bibliographystyle{IEEEbib}
\bibliography{bibl_conf}

\end{document}

